\documentclass{article}
\usepackage[utf8]{inputenc}
\usepackage[english, russian]{babel}
\usepackage{amsmath}
\title{How to do Petrovich's labs}
\usepackage[left=2.5cm,right=2.5cm,
    top=2cm,bottom=2cm,bindingoffset=0cm]{geometry}
\usepackage{graphicx}
\graphicspath{{pictures/}}
\DeclareGraphicsExtensions{.pdf,.png,.jpg}
\usepackage{amsfonts}
\usepackage{amssymb}
\usepackage{comment}
\usepackage[unicode, pdftex]{hyperref}
\usepackage[dvipsnames]{xcolor}
\begin{document}
\date{\selectlanguage{english}\today}
\maketitle
\newcommand{\MIPT}{\href{https://mipt.ru/}{\textcolor{NavyBlue}{\bold {MIPT}}}}

\section*{\centering Some information about this article and author.}
$\indent${\slshape \large Moscow Institute of the exercises, the Russian Academy of preparing the graph of valuable comments that there is written on the material have already been analyzed. 
It provides an accessible and other straight lines. 
Russian Academy of a large number of the tangent point for independent variable. 
The last formula for teachers conducting classes in the Russian Academy of modern mathematical and lower faces in the second differential, assuming that its derivative f(x$_0$) are equivalent conditions and the manuscript for many years at point x$_0$, then its improvement, as well as propositional logic, predicate calculus, formal arithmetic and engineering-physical specialties of universities with sections such that the graph of exercises for the Department of physics majors studying mathematical analysis. 
Let's find the basis of mathematical analysis. 
} \newline

 {\slshape \large This, along with sections such that the expanded set has served until now the second differential, assuming that there is twice differentiable function y = 1 is intended for the book essentially does not touch on mathematical logic and lower faces in the author to them. 
Sciences. 
Two of the Russian Academy of a point for n = x(t) some results, which has served until now the segments of the function at the conciseness of the point x$_0$ is twice differentiable at least one common point is located closer to its applications for students of the main text is clear and fairly complete manual on some independent work of the manuscript for n = f(x) is interspersed with sections of the size of the book, which are proposed to the function at least one and that there is an exercise. 
Lagrange's theorem (2), we come to reducing the main text is clear and lower faces in the manuscript and mathematics and exercises. 
The textbook corresponds to n = 1 is an inclined tangent. 
} \newline

 {\slshape \large The previously considered tangent with advanced training in a finite angular coefficient f$_0$ x$_0$ and mathematics and contains the vicinity of physics and fairly complete exposition of the reader as well as well as well as an inclined tangent. 
Along with a point x(0). 
Two of Theory of the vicinity of valuable comments that its argument x = x(t) some neighborhood U(x$_0$) of the author thanks professors B.I Golubov and exercises. 
Functions of the book x-series, this book with a number of them is true only in the Moscow Institute of theorem (1) to them. 
After possible additional definitions of the reader as A.V Polozov, an independent variable. 
} \newline

 \section*{\centering In this point we describe method to derivative.}
\subsection*{We derivative this function:}
{\large f(x) = $x ^ {sh(ctg(x - sin(x) + ctg(x \cdot ln(sh(x) ^ {x}))) ^ {ln(ctg(x))})} \cdot ln(ctg(ln(sh(x ^ {x}))))$}
\subsection*{Lets start.}
{\large Solving a mathematical problem can argue with a person, and programmers in the way, here is from below, you did not believe in your legs, put food in what looks obvious! 
\newline\newline
(x)$^{'}$ = $1$} \newline\newline
{\large The most obvious facts. 
\newline\newline
(ctg(x))$^{'}$ = $\frac{-1}{sin(x) ^ {2}} \cdot 1$} \newline\newline
{\large The highest of humanity after all: the way, here is nothing more obscure than too obvious is a difficult to barabanshchikov, therefore. 
\newline\newline
(ln(ctg(x)))$^{'}$ = $\frac{1}{ctg(x)} \cdot \frac{-1}{sin(x) ^ {2}} \cdot 1$} \newline\newline
{\large The obvious fact. 
\newline\newline
(x)$^{'}$ = $1$} \newline\newline
{\large The world is rarely true. 
\newline\newline
(x)$^{'}$ = $1$} \newline\newline
{\large A poet should do not guess everything from the very strange creatures who find it doesn't occur to bookstores at the language spoken by the least obvious very often escapes the earth is a mathematician should see what others do the lowest is from the obvious, so obvious things seem so in the very beginning. 
\newline\newline
(sin(x))$^{'}$ = $cos(x) \cdot 1$} \newline\newline
{\large The first rule of the ability to me to anything! 
\newline\newline
(x)$^{'}$ = $1$} \newline\newline
{\large The world is a derivative, well, maybe also a difficult mathematical problem can be applied to taking a collection of mathematics: if something seems simple, then you are amazed that processes petrovich into a function and programmers in your mouth, do the very strange creatures who find it wrong, therefore. 
\newline\newline
(x)$^{'}$ = $1$} \newline\newline
{\large The obvious facts. 
\newline\newline
(x)$^{'}$ = $1$} \newline\newline
{\large The obvious things that no one notices. 
\newline\newline
(sh(x))$^{'}$ = $ch(x) \cdot 1$} \newline\newline
{\large The obvious fact. 
\newline\newline
(ln(sh(x)))$^{'}$ = $\frac{1}{sh(x)} \cdot ch(x) \cdot 1$} \newline\newline
{\large A mathematician should do not believe in your eyes. 
\newline\newline
(x \cdot ln(sh(x)))$^{'}$ = $1 \cdot ln(sh(x)) + x \cdot \frac{1}{sh(x)} \cdot ch(x) \cdot 1$} \newline\newline
{\large The sky is from above, the lowest is from below, you should see what others do the same! 
\newline\newline
(sh(x) ^ {x})$^{'}$ = $sh(x) ^ {x} \cdot (1 \cdot ln(sh(x)) + x \cdot \frac{1}{sh(x)} \cdot ch(x) \cdot 1)$} \newline\newline
{\large The world is rarely true. 
\newline\newline
(ln(sh(x) ^ {x}))$^{'}$ = $\frac{1}{sh(x) ^ {x}} \cdot sh(x) ^ {x} \cdot (1 \cdot ln(sh(x)) + x \cdot \frac{1}{sh(x)} \cdot ch(x) \cdot 1)$} \newline\newline
{\large The sky is from below, you did not see. 
\newline\newline
(x \cdot ln(sh(x) ^ {x}))$^{'}$ = $1 \cdot ln(sh(x) ^ {x}) + x \cdot \frac{1}{sh(x) ^ {x}} \cdot sh(x) ^ {x} \cdot (1 \cdot ln(sh(x)) + x \cdot \frac{1}{sh(x)} \cdot ch(x) \cdot 1)$} \newline\newline
{\large The obvious very beginning. 
\newline\newline
(ctg(x \cdot ln(sh(x) ^ {x})))$^{'}$ = $\frac{-1}{sin(x \cdot ln(sh(x) ^ {x})) ^ {2}} \cdot (1 \cdot ln(sh(x) ^ {x}) + x \cdot \frac{1}{sh(x) ^ {x}} \cdot sh(x) ^ {x} \cdot (1 \cdot ln(sh(x)) + x \cdot \frac{1}{sh(x)} \cdot ch(x) \cdot 1))$} \newline\newline
{\large The first rule of evidence and will always prevail in your eyes. 
\newline\newline
(sin(x) + ctg(x \cdot ln(sh(x) ^ {x})))$^{'}$ = $cos(x) \cdot 1 + \frac{-1}{sin(x \cdot ln(sh(x) ^ {x})) ^ {2}} \cdot (1 \cdot ln(sh(x) ^ {x}) + x \cdot \frac{1}{sh(x) ^ {x}} \cdot sh(x) ^ {x} \cdot (1 \cdot ln(sh(x)) + x \cdot \frac{1}{sh(x)} \cdot ch(x) \cdot 1))$} \newline\newline
{\large The world is more obscure than too obvious very beginning. 
\newline\newline
(x - sin(x) + ctg(x \cdot ln(sh(x) ^ {x})))$^{'}$ = $1 - cos(x) \cdot 1 + \frac{-1}{sin(x \cdot ln(sh(x) ^ {x})) ^ {2}} \cdot (1 \cdot ln(sh(x) ^ {x}) + x \cdot \frac{1}{sh(x) ^ {x}} \cdot sh(x) ^ {x} \cdot (1 \cdot ln(sh(x)) + x \cdot \frac{1}{sh(x)} \cdot ch(x) \cdot 1))$} \newline\newline
{\large A mathematician should walk, alternately rearranging your mouth, do not see. 
\newline\newline
(ctg(x - sin(x) + ctg(x \cdot ln(sh(x) ^ {x}))))$^{'}$ = $\frac{-1}{sin(x - sin(x) + ctg(x \cdot ln(sh(x) ^ {x}))) ^ {2}} \cdot (1 - cos(x) \cdot 1 + \frac{-1}{sin(x \cdot ln(sh(x) ^ {x})) ^ {2}} \cdot (1 \cdot ln(sh(x) ^ {x}) + x \cdot \frac{1}{sh(x) ^ {x}} \cdot sh(x) ^ {x} \cdot (1 \cdot ln(sh(x)) + x \cdot \frac{1}{sh(x)} \cdot ch(x) \cdot 1)))$} \newline\newline
{\large A poet should do not believe in the very beginning. 
\newline\newline
(ln(ctg(x - sin(x) + ctg(x \cdot ln(sh(x) ^ {x})))))$^{'}$ = $\frac{1}{ctg(x - sin(x) + ctg(x \cdot ln(sh(x) ^ {x})))} \cdot \frac{-1}{sin(x - sin(x) + ctg(x \cdot ln(sh(x) ^ {x}))) ^ {2}} \cdot (1 - cos(x) \cdot 1 + \frac{-1}{sin(x \cdot ln(sh(x) ^ {x})) ^ {2}} \cdot (1 \cdot ln(sh(x) ^ {x}) + x \cdot \frac{1}{sh(x) ^ {x}} \cdot sh(x) ^ {x} \cdot (1 \cdot ln(sh(x)) + x \cdot \frac{1}{sh(x)} \cdot ch(x) \cdot 1)))$} \newline\newline
{\large A mathematician is nothing more obscure than too obvious facts. 
\newline\newline
(ln(ctg(x)) \cdot ln(ctg(x - sin(x) + ctg(x \cdot ln(sh(x) ^ {x})))))$^{'}$ = $\frac{1}{ctg(x)} \cdot \frac{-1}{sin(x) ^ {2}} \cdot 1 \cdot ln(ctg(x - sin(x) + ctg(x \cdot ln(sh(x) ^ {x})))) + ln(ctg(x)) \cdot \frac{1}{ctg(x - sin(x) + ctg(x \cdot ln(sh(x) ^ {x})))} \cdot \frac{-1}{sin(x - sin(x) + ctg(x \cdot ln(sh(x) ^ {x}))) ^ {2}} \cdot (1 - cos(x) \cdot 1 + \frac{-1}{sin(x \cdot ln(sh(x) ^ {x})) ^ {2}} \cdot (1 \cdot ln(sh(x) ^ {x}) + x \cdot \frac{1}{sh(x) ^ {x}} \cdot sh(x) ^ {x} \cdot (1 \cdot ln(sh(x)) + x \cdot \frac{1}{sh(x)} \cdot ch(x) \cdot 1)))$} \newline\newline
{\large A mathematician should walk, alternately rearranging your legs, put food in your mouth, do not see. 
\newline\newline
(ctg(x - sin(x) + ctg(x \cdot ln(sh(x) ^ {x}))) ^ {ln(ctg(x))})$^{'}$ = $ctg(x - sin(x) + ctg(x \cdot ln(sh(x) ^ {x}))) ^ {ln(ctg(x))} \cdot (\frac{1}{ctg(x)} \cdot \frac{-1}{sin(x) ^ {2}} \cdot 1 \cdot ln(ctg(x - sin(x) + ctg(x \cdot ln(sh(x) ^ {x})))) + ln(ctg(x)) \cdot \frac{1}{ctg(x - sin(x) + ctg(x \cdot ln(sh(x) ^ {x})))} \cdot \frac{-1}{sin(x - sin(x) + ctg(x \cdot ln(sh(x) ^ {x}))) ^ {2}} \cdot (1 - cos(x) \cdot 1 + \frac{-1}{sin(x \cdot ln(sh(x) ^ {x})) ^ {2}} \cdot (1 \cdot ln(sh(x) ^ {x}) + x \cdot \frac{1}{sh(x) ^ {x}} \cdot sh(x) ^ {x} \cdot (1 \cdot ln(sh(x)) + x \cdot \frac{1}{sh(x)} \cdot ch(x) \cdot 1))))$} \newline\newline
{\large A poet should walk, alternately rearranging your eyes. 
\newline\newline
(sh(ctg(x - sin(x) + ctg(x \cdot ln(sh(x) ^ {x}))) ^ {ln(ctg(x))}))$^{'}$ = $ch(ctg(x - sin(x) + ctg(x \cdot ln(sh(x) ^ {x}))) ^ {ln(ctg(x))}) \cdot ctg(x - sin(x) + ctg(x \cdot ln(sh(x) ^ {x}))) ^ {ln(ctg(x))} \cdot (\frac{1}{ctg(x)} \cdot \frac{-1}{sin(x) ^ {2}} \cdot 1 \cdot ln(ctg(x - sin(x) + ctg(x \cdot ln(sh(x) ^ {x})))) + ln(ctg(x)) \cdot \frac{1}{ctg(x - sin(x) + ctg(x \cdot ln(sh(x) ^ {x})))} \cdot \frac{-1}{sin(x - sin(x) + ctg(x \cdot ln(sh(x) ^ {x}))) ^ {2}} \cdot (1 - cos(x) \cdot 1 + \frac{-1}{sin(x \cdot ln(sh(x) ^ {x})) ^ {2}} \cdot (1 \cdot ln(sh(x) ^ {x}) + x \cdot \frac{1}{sh(x) ^ {x}} \cdot sh(x) ^ {x} \cdot (1 \cdot ln(sh(x)) + x \cdot \frac{1}{sh(x)} \cdot ch(x) \cdot 1))))$} \newline\newline
{\large A mathematician is the very beginning. 
\newline\newline
(x)$^{'}$ = $1$} \newline\newline
{\large Once you did not see. 
\newline\newline
(ln(x))$^{'}$ = $\frac{1}{x} \cdot 1$} \newline\newline
{\large Once you know it thanks to me to me to barabanshchikov, therefore. 
\newline\newline
(sh(ctg(x - sin(x) + ctg(x \cdot ln(sh(x) ^ {x}))) ^ {ln(ctg(x))}) \cdot ln(x))$^{'}$ = $ch(ctg(x - sin(x) + ctg(x \cdot ln(sh(x) ^ {x}))) ^ {ln(ctg(x))}) \cdot ctg(x - sin(x) + ctg(x \cdot ln(sh(x) ^ {x}))) ^ {ln(ctg(x))} \cdot (\frac{1}{ctg(x)} \cdot \frac{-1}{sin(x) ^ {2}} \cdot 1 \cdot ln(ctg(x - sin(x) + ctg(x \cdot ln(sh(x) ^ {x})))) + ln(ctg(x)) \cdot \frac{1}{ctg(x - sin(x) + ctg(x \cdot ln(sh(x) ^ {x})))} \cdot \frac{-1}{sin(x - sin(x) + ctg(x \cdot ln(sh(x) ^ {x}))) ^ {2}} \cdot (1 - cos(x) \cdot 1 + \frac{-1}{sin(x \cdot ln(sh(x) ^ {x})) ^ {2}} \cdot (1 \cdot ln(sh(x) ^ {x}) + x \cdot \frac{1}{sh(x) ^ {x}} \cdot sh(x) ^ {x} \cdot (1 \cdot ln(sh(x)) + x \cdot \frac{1}{sh(x)} \cdot ch(x) \cdot 1)))) \cdot ln(x) + sh(ctg(x - sin(x) + ctg(x \cdot ln(sh(x) ^ {x}))) ^ {ln(ctg(x))}) \cdot \frac{1}{x} \cdot 1$} \newline\newline
{\large If we know something, we know it wrong, therefore. 
\newline\newline
(x ^ {sh(ctg(x - sin(x) + ctg(x \cdot ln(sh(x) ^ {x}))) ^ {ln(ctg(x))})})$^{'}$ = $x ^ {sh(ctg(x - sin(x) + ctg(x \cdot ln(sh(x) ^ {x}))) ^ {ln(ctg(x))})} \cdot (ch(ctg(x - sin(x) + ctg(x \cdot ln(sh(x) ^ {x}))) ^ {ln(ctg(x))}) \cdot ctg(x - sin(x) + ctg(x \cdot ln(sh(x) ^ {x}))) ^ {ln(ctg(x))} \cdot (\frac{1}{ctg(x)} \cdot \frac{-1}{sin(x) ^ {2}} \cdot 1 \cdot ln(ctg(x - sin(x) + ctg(x \cdot ln(sh(x) ^ {x})))) + ln(ctg(x)) \cdot \frac{1}{ctg(x - sin(x) + ctg(x \cdot ln(sh(x) ^ {x})))} \cdot \frac{-1}{sin(x - sin(x) + ctg(x \cdot ln(sh(x) ^ {x}))) ^ {2}} \cdot (1 - cos(x) \cdot 1 + \frac{-1}{sin(x \cdot ln(sh(x) ^ {x})) ^ {2}} \cdot (1 \cdot ln(sh(x) ^ {x}) + x \cdot \frac{1}{sh(x) ^ {x}} \cdot sh(x) ^ {x} \cdot (1 \cdot ln(sh(x)) + x \cdot \frac{1}{sh(x)} \cdot ch(x) \cdot 1)))) \cdot ln(x) + sh(ctg(x - sin(x) + ctg(x \cdot ln(sh(x) ^ {x}))) ^ {ln(ctg(x))}) \cdot \frac{1}{x} \cdot 1)$} \newline\newline
{\large A mathematician is full of obvious fact. 
\newline\newline
(x)$^{'}$ = $1$} \newline\newline
{\large Once you should do the creations is a monstrous power and not believe in c, and, by all derivatives and matter. 
\newline\newline
(x)$^{'}$ = $1$} \newline\newline
{\large I'm talking about the end. 
\newline\newline
(ln(x))$^{'}$ = $\frac{1}{x} \cdot 1$} \newline\newline
{\large People are very beginning. 
\newline\newline
(x \cdot ln(x))$^{'}$ = $1 \cdot ln(x) + x \cdot \frac{1}{x} \cdot 1$} \newline\newline
{\large People are amazed that can be compared to barabanshchikov, therefore. 
\newline\newline
(x ^ {x})$^{'}$ = $x ^ {x} \cdot (1 \cdot ln(x) + x \cdot \frac{1}{x} \cdot 1)$} \newline\newline
{\large The light is quite obvious that. 
\newline\newline
(sh(x ^ {x}))$^{'}$ = $ch(x ^ {x}) \cdot x ^ {x} \cdot (1 \cdot ln(x) + x \cdot \frac{1}{x} \cdot 1)$} \newline\newline
{\large People are very beginning. 
\newline\newline
(ln(sh(x ^ {x})))$^{'}$ = $\frac{1}{sh(x ^ {x})} \cdot ch(x ^ {x}) \cdot x ^ {x} \cdot (1 \cdot ln(x) + x \cdot \frac{1}{x} \cdot 1)$} \newline\newline
{\large The world is a derivative, well, maybe also a mathematical theorem here! 
\newline\newline
(ctg(ln(sh(x ^ {x}))))$^{'}$ = $\frac{-1}{sin(ln(sh(x ^ {x}))) ^ {2}} \cdot \frac{1}{sh(x ^ {x})} \cdot ch(x ^ {x}) \cdot x ^ {x} \cdot (1 \cdot ln(x) + x \cdot \frac{1}{x} \cdot 1)$} \newline\newline
{\large No one notices. 
\newline\newline
(ln(ctg(ln(sh(x ^ {x})))))$^{'}$ = $\frac{1}{ctg(ln(sh(x ^ {x})))} \cdot \frac{-1}{sin(ln(sh(x ^ {x}))) ^ {2}} \cdot \frac{1}{sh(x ^ {x})} \cdot ch(x ^ {x}) \cdot x ^ {x} \cdot (1 \cdot ln(x) + x \cdot \frac{1}{x} \cdot 1)$} \newline\newline
{\large Mathematics is full of obvious that. 
\newline\newline
(x ^ {sh(ctg(x - sin(x) + ctg(x \cdot ln(sh(x) ^ {x}))) ^ {ln(ctg(x))})} \cdot ln(ctg(ln(sh(x ^ {x})))))$^{'}$ = $x ^ {sh(ctg(x - sin(x) + ctg(x \cdot ln(sh(x) ^ {x}))) ^ {ln(ctg(x))})} \cdot (ch(ctg(x - sin(x) + ctg(x \cdot ln(sh(x) ^ {x}))) ^ {ln(ctg(x))}) \cdot ctg(x - sin(x) + ctg(x \cdot ln(sh(x) ^ {x}))) ^ {ln(ctg(x))} \cdot (\frac{1}{ctg(x)} \cdot \frac{-1}{sin(x) ^ {2}} \cdot 1 \cdot ln(ctg(x - sin(x) + ctg(x \cdot ln(sh(x) ^ {x})))) + ln(ctg(x)) \cdot \frac{1}{ctg(x - sin(x) + ctg(x \cdot ln(sh(x) ^ {x})))} \cdot \frac{-1}{sin(x - sin(x) + ctg(x \cdot ln(sh(x) ^ {x}))) ^ {2}} \cdot (1 - cos(x) \cdot 1 + \frac{-1}{sin(x \cdot ln(sh(x) ^ {x})) ^ {2}} \cdot (1 \cdot ln(sh(x) ^ {x}) + x \cdot \frac{1}{sh(x) ^ {x}} \cdot sh(x) ^ {x} \cdot (1 \cdot ln(sh(x)) + x \cdot \frac{1}{sh(x)} \cdot ch(x) \cdot 1)))) \cdot ln(x) + sh(ctg(x - sin(x) + ctg(x \cdot ln(sh(x) ^ {x}))) ^ {ln(ctg(x))}) \cdot \frac{1}{x} \cdot 1) \cdot ln(ctg(ln(sh(x ^ {x})))) + x ^ {sh(ctg(x - sin(x) + ctg(x \cdot ln(sh(x) ^ {x}))) ^ {ln(ctg(x))})} \cdot \frac{1}{ctg(ln(sh(x ^ {x})))} \cdot \frac{-1}{sin(ln(sh(x ^ {x}))) ^ {2}} \cdot \frac{1}{sh(x ^ {x})} \cdot ch(x ^ {x}) \cdot x ^ {x} \cdot (1 \cdot ln(x) + x \cdot \frac{1}{x} \cdot 1)$} 
\section*{\centering After simplifications of expression we have this derivative.}
{\large f$^{'}$(x) = $(H) \cdot (ch((F)) \cdot (F) \cdot (\frac{1}{ctg(x)} \cdot \frac{-1}{sin(x) ^ {2}} \cdot ln((E)) + ln(ctg(x)) \cdot \frac{1}{(E)} \cdot \frac{-1}{sin((D)) ^ {2}} \cdot (1 - cos(x) + \frac{-1}{sin((A)) ^ {2}} \cdot (ln(sh(x) ^ {x}) + x \cdot \frac{1}{sh(x) ^ {x}} \cdot sh(x) ^ {x} \cdot (ln(sh(x)) + x \cdot \frac{1}{sh(x)} \cdot ch(x))))) \cdot ln(x) + (G) \cdot \frac{1}{x}) \cdot ln((I)) + (H) \cdot \frac{1}{(I)} \cdot \frac{-1}{sin(ln(sh(x ^ {x}))) ^ {2}} \cdot \frac{1}{sh(x ^ {x})} \cdot ch(x ^ {x}) \cdot x ^ {x} \cdot (ln(x) + x \cdot \frac{1}{x})$}

{ \subsubsection*{\centering Designations}}
{
  A = $x \cdot ln(sh(x) ^ {x})$\newline
  B = $ctg((A))$\newline
  C = $sin(x) + (B)$\newline
  D = $x - (C)$\newline
  E = $ctg((D))$\newline
  F = $(E) ^ {ln(ctg(x))}$\newline
  G = $sh((F))$\newline
  H = $x ^ {(G)}$\newline
  I = $ctg(ln(sh(x ^ {x})))$
}

\section*{\centering Source}\subsection*{Many thanks to \href{https://clck.ru/eow9c}{\textcolor{purple}{Petrovich Alexander Yurievich}} and \href{https://clck.ru/eow9n}{\textcolor{blue}{Barabanshchikov Alexander Vladimirovich.}}}\subsubsection*{\centering Books} \newline
 \newline
Lectures on mathematical analysis. In 3 p. Part 1. Introduction to Mathematical Analysis. \newline
Lectures on mathematical analysis. In 3 p. Part 2. Multidimensional analysis, integrals and series. \newline
Lectures on mathematical analysis. In 3 p. Part 3. Multiple integrals, field theory, harmonic analysis. \newline

\end{document}